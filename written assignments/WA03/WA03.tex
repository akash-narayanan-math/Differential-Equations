\documentclass[11pt, titlepage]{article}
% Common packages/environments to remove clutter

% Packages
\usepackage[utf8]{inputenc}
\usepackage{amsmath, amsfonts, amssymb, amsthm, enumitem, tikz, import, mathtools}
\usepackage[
  top=2cm,
  bottom=2cm,
  left=3cm,
  right=3cm,
  headheight=17pt,
  includehead, includefoot,
  heightrounded,
]{geometry}

% Problem environment
\newtheoremstyle{emptyplain}
    {}          % default space above
    {}          % default space below
    {}          % default body font
    {}          % no indent
    {\bfseries} % head font
    {.}         % punctuation after theorem head
    { }         % space after theorem head
    {#3}
\theoremstyle{emptyplain}
\newtheorem*{problem}{}

% Solution Environment
\newenvironment{solution}{
  \begin{proof}[Solution]
    \vspace{-2px}
    \setlength{\parskip}{4px}
    \setlength{\parindent}{0px}
}{
\end{proof}
}

\usepackage{caption, subcaption}

% Opening
\title{Math 2552 Written HW Set 3}
\author{Akash Narayanan}
\date{February 9, 2021}

\begin{document}
  \maketitle

  % Judson 3.2 #11
  \begin{problem}[Judson 3.2.11]
    Consider the system
    \begin{align*}
      x' &= ax + y \\
      y' &= 2ax + 2y,
    \end{align*}
    where \(a \in \mathbb{R}\).
    For what values of \(a\) do you find a bifurcation
    (a change in the type of phase portrait)?
    Sketch typical phase portraits for a value of \(a\) above and below the bifurcation point.
  \end{problem}

  \begin{solution}
    We start by rewriting the system in matrix form \(\vec{x} \,' = A \vec{x}\) where
    \begin{equation*}
      A =
      \begin{pmatrix}
        a & 1 \\
        2a & 2
      \end{pmatrix}
    \end{equation*}
    The characteristic polynomial is \(\det (A - \lambda I) = \lambda^{2} - (a + 2) \lambda\).
    Setting this equal to 0 and solving for \(\lambda\) yields the eigenvalues \(\lambda_{1} = 0\) and \(\lambda_{2} = a + 2\).
    From here it is clear that there is a bifurcation at \(a = -2\), where \(\lambda_{2}\) changes sign.

    By solving the system \((A - \lambda I)\vec{v} = \vec{0}\), we obtain the following eigenvectors:
    \begin{align*}
      \vec{v}_{1} =
      \begin{pmatrix}
        1 \\
        -a
      \end{pmatrix}
      &&
      \vec{v}_{2} =
      \begin{pmatrix}
        1 \\
        2
      \end{pmatrix}
    \end{align*}
    Note that the eigenvalue associated with \(\vec{v}_{1}\) is 0.
    That is, the points along this line are equilibrium solutions to the system of equations.

    We have that the general solution to the system is
    \begin{equation*}
      \vec{x} \,' = c_{1}
      \begin{pmatrix}
        1 \\
        -a
      \end{pmatrix}
      + c_{2} e^{(a + 2) t}
      \begin{pmatrix}
        1 \\
        2
      \end{pmatrix}
    \end{equation*}
    Now we draw two phase portraits for values of \(a\) above and below -2.
    \begin{figure}[h]
      \centering
      \begin{subfigure}[b]{0.35\textwidth}
        \centering
        \def\svgwidth{0.9\columnwidth}
        \import{media/}{phasePortrait1.pdf_tex}
        \captionsetup{labelformat=empty}
        \caption{\(a = 1\)}
      \end{subfigure}
      \begin{subfigure}[b]{0.35\textwidth}
        \centering
        \def\svgwidth{0.9\columnwidth}
        \import{media/}{phasePortrait2.pdf_tex}
        \captionsetup{labelformat=empty}
        \caption{\(a = -4\)}
      \end{subfigure}
    \end{figure}

  \end{solution}

  \pagebreak

  % Judson 3.4 #3
  \begin{problem}[Judson 3.4.3]
    Find the general solution of the linear system shown below and give a sketch of the phase portrait with a few solution curves.
    \begin{align*}
      x' &= -x - 4y \\
      y' &= 3x - 2y
    \end{align*}
  \end{problem}

  \begin{solution}
    We rewrite the system in matrix form \(\vec{x} \,' = A \vec{x}\) where
    \begin{equation*}
      A =
      \begin{pmatrix}
        -1 & -4 \\
        3 & -2
      \end{pmatrix}
    \end{equation*}
    This matrix has the eigenvalues \(\lambda_{1, 2} = -3 / 2 \pm i \sqrt{47} / 2\).
    % \begin{equation*}
    %   \lambda_{1, 2} = -\frac{3}{2} \pm \frac{\sqrt{47}}{2} i
    % \end{equation*}
    The corresponding eigenvectors are
    \begin{align*}
      \vec{v}_{1} =
      \begin{pmatrix}
        1 + i \sqrt{47} \\
        6
      \end{pmatrix}
      &&
      \vec{v}_{2} =
      \begin{pmatrix}
        1 - i \sqrt{47} \\
        6
      \end{pmatrix}
    \end{align*}
    Now we can define \(\vec{w}_{1} = e^{\lambda_{1}t} \vec{v}_{1}\). Expanding this and using Euler's formula yields
    \begin{align*}
      \vec{w}_{1}
      % &= e^{-\frac{3}{2}t} e^{i \frac{\sqrt{47}}{2}t}
      % \begin{pmatrix}
      %   1 + i \sqrt{47} \\
      %   6
      % \end{pmatrix} \\
      &= e^{-\frac{3}{2}t} \left( \cos \left(\frac{\sqrt{47}}{2}t \right) + i \sin \left( \frac{\sqrt{47}}{2}t \right)\right)
      \begin{pmatrix}
        1 + i \sqrt{47} \\
        6
      \end{pmatrix} \\
      &= e^{-\frac{3}{2}t}
      \begin{pmatrix}
        \cos(\frac{\sqrt{47}}{2} t) - \sqrt{47} \sin(\frac{\sqrt{47}}{2} t) \\
        6 \cos(\frac{\sqrt{47}}{2} t)
      \end{pmatrix}
      + i e^{-\frac{3}{2} t}
      \begin{pmatrix}
        \sqrt{47} \cos(\frac{\sqrt{47}}{2} t) + \sin(\frac{\sqrt{47}}{2} t) \\
        6 \sin(\frac{\sqrt{47}}{2} t)
      \end{pmatrix}
    \end{align*}
    Then the general solution is
    \begin{equation*}
      \vec{x}(t) = c_{1} e^{-\frac{3}{2} t}
      \begin{pmatrix}
        \cos(\frac{\sqrt{47}}{2} t) - \sqrt{47} \sin(\frac{\sqrt{47}}{2} t) \\
        6 \cos(\frac{\sqrt{47}}{2} t)
      \end{pmatrix} +
      c_{2} e^{-\frac{3}{2} t}
      \begin{pmatrix}
        \sqrt{47} \cos(\frac{\sqrt{47}}{2} t) + \sin(\frac{\sqrt{47}}{2} t) \\
        6 \sin(\frac{\sqrt{47}}{2} t)
      \end{pmatrix}
    \end{equation*}

    \begin{figure}[h]
      \centering
      \def\svgwidth{0.3\columnwidth}
      \import{media/}{phasePortrait3.pdf_tex}
      \captionsetup{labelformat=empty}
      \caption{A phase portrait for the linear system}
    \end{figure}
  \end{solution}

  \pagebreak

  % Judson 3.5 #7
  \begin{problem}[Judson 3.5.7]
    Solve the following linear system for the given initial values and give a sketch of the phase portrait with a few solution curves.
    \begin{align*}
      x' &= 9x + 4y \\
      y' &= -9x - 3y \\
      x(0) &= 2 \\
      y(0) &= -3
    \end{align*}
  \end{problem}

  \begin{solution}
    We start by rewriting the system in matrix form \(\vec{x} \,' = A \vec{x}\) where
    \begin{equation*}
      A =
      \begin{pmatrix}
        9 & 4 \\
        -9 & - 3
      \end{pmatrix}
    \end{equation*}
    The eigenvalue is \(\lambda = 3\) with multiplicity 2.
    The corresponding eigenvector is
    \begin{equation*}
      \vec{v} =
      \begin{pmatrix}
        -2 \\
        3
      \end{pmatrix}
    \end{equation*}
    \begin{figure}[h]
      \centering
      \def\svgwidth{0.5\columnwidth}
      \import{media/}{phasePortrait4.pdf_tex}
      \captionsetup{labelformat=empty}
      \caption{A sketch of the solution curve}
    \end{figure}
  \end{solution}
\end{document}
