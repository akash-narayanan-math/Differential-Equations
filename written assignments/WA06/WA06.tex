\documentclass[11pt, titlepage]{article}
% Common packages/environments to remove clutter

% Packages
\usepackage[utf8]{inputenc}
\usepackage{amsmath, amsfonts, amssymb, amsthm, enumitem, tikz, import, mathtools}
\usepackage[
  top=2cm,
  bottom=2cm,
  left=3cm,
  right=3cm,
  headheight=17pt,
  includehead, includefoot,
  heightrounded,
]{geometry}

% Problem environment
\newtheoremstyle{emptyplain}
    {}          % default space above
    {}          % default space below
    {}          % default body font
    {}          % no indent
    {\bfseries} % head font
    {.}         % punctuation after theorem head
    { }         % space after theorem head
    {#3}
\theoremstyle{emptyplain}
\newtheorem*{problem}{}

% Solution Environment
\newenvironment{solution}{
  \begin{proof}[Solution]
    \vspace{-2px}
    \setlength{\parskip}{4px}
    \setlength{\parindent}{0px}
}{
\end{proof}
}


% Opening
\title{Math 2552 Written HW Set 6}
\author{Akash Narayanan}
\date{March 9, 2021}

\begin{document}
    \maketitle

    % Trench 5.4 #7
    \begin{problem}[Trench 5.4.7]
        Find a particular solution.
        \[
        y'' - 4y' - 5y = -6 x e^{-x}
        \] 
    \end{problem}

    \begin{solution}
        We find a particular solution using the method of undetermined
        coefficients. First, we will find the general solution to the homogeneous
        equation. Note that its characteristic equation is $r^2 - 4r - 5 =
        (r - 5)(r + 1)$ so the general solution to the homogeneous equation is 
        \[
        y_h = c_1 e^{5x} + c_2 e^{-x}.
        \] 
        A standard first guess for the particular solution would be $y_p = (ax +
        b) e^{-x}$. However, note that this particular solution contains a term
        that is in the general solution to the homogeneous equation. This is
        problematic because it effectively means that attempting to solve for
        $a$ and $b$ by plugging the solution into the differential equation
        would only yield one equation despite having 2 variables. We remedy this
        by including a factor of $x$. This yields
        \begin{align*}
            y_p &= (ax + b) x e^{-x} \\
            y_p' &= -(ax^2 + (b - 2a)x - b) e^{-x} \\
            y_p'' &= (ax^2 + (b - 4a) x - 2b + 2a) e^{-x}.
        \end{align*}
        Plugging this into the differential equation yields (after a fair bit of
        simplification)
        \begin{gather*}
            (ax^2 + (b - 4a) x - 2b + 2a) e^{-x} + 4 (ax^2 + (b - 2a)x - b)
            e^{-x} - 5 (ax + b) x e^{-x} = -6xe^{-x} \\
            \Longrightarrow (-12ax + 2a - 6b) e^{-x} = -6xe^{-x}
        \end{gather*}
        Equating terms of equal degree in the polynomial factor, we obtain two
        equations:
        \begin{gather*}
            -12a = -6 \\
            2a - 6b = 0
        \end{gather*}
        This has the solution $a = 1 / 2$ and $b = 1 / 6$. Thus, a particular
        solution to the differential equation (after distributing the $x$ to the
        linear equation) is
        \[
            y_p = (\frac{1}{2} x^2 + \frac{1}{6} x) e^{-x}.
        \] 
    \end{solution}

    \pagebreak


    % Trench 5.7 #7
    \begin{problem}[Trench 5.7.7]
        Use variation of parameters to find a particular solution, given the
        solutions $y_1, y_2$ of the complementary equation.
        \[
        x^2 y'' + xy' - y = 2 x^2 + 2; \quad y_1 = x, \quad y_2 = \frac{1}{x}
        \] 
    \end{problem}

    \begin{solution}
        Before starting, we will rewrite the differential equation so that
        $y''$ has coefficient 1:
        \begin{equation}
            y'' + \frac{1}{x}y - \frac{1}{x^2}y = 2 + \frac{2}{x^2}
        \end{equation}
        Given the fundamental set of solutions to the homogeneous equation, we
        construct a particular solution of the form
        \[
        y_p = u_1 y_1 + u_2 y_2
        \] 
        for function $u_1$ and $u_2$. We start by calculating the Wronskian
        \[
            W(y_1, y_2) = \det
            \begin{pmatrix}
                y_1 & y_2 \\
                y_1' & y_2'
            \end{pmatrix} = \det
            \begin{pmatrix}
                x & x^{-1} \\
                1 & -x^{-2}
            \end{pmatrix} = 
            -\frac{2}{x}.
        \] 
        Then we have  
        \begin{align*}
            u_1'(x) &= \frac{-y_2(x) f(x)}{W(y_1, y_2)(x)} =
            \frac{-x^{-1}(2+2x^{-2})}{-2x^{-1}} = 1 + \frac{1}{x^2} \\
            u_2'(x) &= \frac{y_1(x) f(x)}{W(y_1, y_2)(x)} = \frac{x
            (2+2x^{-2})}{-2x^{-1}} = -x^2 - 1
        \end{align*}
        where $f(x)$ is the right side of (1).
        Integrating both functions with respect to $x$ yields
        \begin{align*}
            u_1 &= \int 1 + \frac{1}{x^2} \, dx = x - \frac{1}{x} \\
            u_2 &= \int -x^2 - 1 \, dx = - \frac{x^{3}}{3} - x
        \end{align*}
        Thus, a particular solution to the differential equation is
        \begin{align*}
            y_p(x) &= \left(x - \frac{1}{x} \right) x + \left( - \frac{x^3}{3} -
            x \right) \frac{1}{x} \\
                   &= \frac{2}{3} x^2 - 2
        \end{align*}
    \end{solution}
    \pagebreak


    % Trench 10.7 #1
    \begin{problem}[Trench 10.7.1]
        Find a particular solution.
        \[
        y' = 
        \begin{bmatrix}
            -1 & -4 \\
            -1 & -1
        \end{bmatrix}
        y +
        \begin{bmatrix}
            21 e^{4t} \\
            8 e^{-3t}
        \end{bmatrix}
        \]  
    \end{problem}

    \begin{solution}
        Let $A$ denote the coefficient matrix of $y$. Then  
        \[
            \det(A - \lambda I) = \lambda^2 + 2\lambda - 3 = (\lambda + 3)
            (\lambda - 1)
        \] 
        so the eigenvalues of the system are $\lambda_1 = 1$ and $\lambda_2 =
        -3$. The corresponding eigenvectors are found by solving $(A - \lambda
        I)\vec{v} = 0$. We have
        \[
            (A - I) \vec{v} = 
        \begin{bmatrix}
            -2 & -4 \\
            -1 & -2
        \end{bmatrix} \vec{v} = 0 \Longrightarrow -x_1 - 2x_2 = 0
        \Longrightarrow -x_1 = 2x_2
        \] 
        Letting $x_2 = 1$, we find the eigenvector corresponding to $\lambda_1 =
        1$
        to be
        \[
            \vec{v}_1 = 
            \begin{pmatrix}
                -2 \\
                1
            \end{pmatrix}
        \] 
        Similarly, we find
        \[
            (A + 3I)\vec{v} = 
            \begin{bmatrix}
                2 & -4 \\
                -1 & 2
            \end{bmatrix} \vec{v} = 0 \Longrightarrow -x_1 + 2x_2 = 0
            \Longrightarrow 2x_2 = x_1
        \] 
        Letting $x_2 = 1$, we find the eigenvector corresponding to $\lambda_2 =
        3$ to be
        \[
            \vec{v}_2 = 
            \begin{pmatrix}
                2 \\
                1
            \end{pmatrix}
        \] 
        Then the general solution to the homogeneous system is
        \[
        y_h = c_1 
        \begin{pmatrix}
            -2e^{t} \\
            e^{t}
        \end{pmatrix} + c_2
        \begin{pmatrix}
            2e^{-3t} \\
            e^{-3t}
        \end{pmatrix}
        \] 
        The corresponding fundamental matrix is
        \[
        Y = 
        \begin{pmatrix}
            -2e^{t} & 2e^{-3t} \\
            e^{t} & e^{-3t}
        \end{pmatrix}
        \] 
        We find
        \begin{gather*}
            \det(Y) = -2e^{-2t} - 2e^{-2t} = -4e^{-2t} \\
            Y^{-1} = \frac{1}{\det(Y)} 
            \begin{pmatrix}
                e^{-3t} & -2e^{-3t} \\
                -e^{t} & -2e^{t}
            \end{pmatrix} = \frac{1}{4}
            \begin{pmatrix}
                -e^{-t} & 2e^{-t} \\
                e^{3t} & 2e^{3t}
            \end{pmatrix}
        \end{gather*}
        Then a particular solution is
        \[
        y_p = Y \int Y^{-1} g \, dt
        \] 
        where $g$ is nonhomogeneous part of the system of differential
        equations. Calculating, we have
        \begin{align*}
            y_p &= \frac{1}{4} Y \int 
            \begin{pmatrix}
                -e^{-t} & 2e^{-t} \\
                e^{3t} & 2e^{3t}
            \end{pmatrix}
            \begin{pmatrix}
                21 e^{4t} \\
                8 e^{-3t}
            \end{pmatrix} \, dt \\
                &= \frac{1}{4} Y \int
                \begin{pmatrix}
                    -21e^{3t} + 16e^{-4t} \\
                    21e^{7t} + 16
                \end{pmatrix} \, dt \\
                &= \frac{1}{4}
                \begin{pmatrix}
                    -2e^{t} & 2e^{-3t} \\
                    e^{t} & e^{-3t}
                \end{pmatrix}
                \begin{pmatrix}
                    -7e^{3t} - 4e^{-4t} \\
                    3e^{7t} + 16t
                \end{pmatrix} \\
                &= 
                \begin{pmatrix}
                    8te^{-3t} + 5e^{4t} + 2e^{-3t} \\
                    4te^{-3t} - e^{4t} - e^{-3t}
                \end{pmatrix}
        \end{align*}
    \end{solution}
\end{document}
