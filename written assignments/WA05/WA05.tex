\documentclass[11pt, titlepage]{article}
% Common packages/environments to remove clutter

% Packages
\usepackage[utf8]{inputenc}
\usepackage{amsmath, amsfonts, amssymb, amsthm, enumitem, tikz, import, mathtools}
\usepackage[
  top=2cm,
  bottom=2cm,
  left=3cm,
  right=3cm,
  headheight=17pt,
  includehead, includefoot,
  heightrounded,
]{geometry}

% Problem environment
\newtheoremstyle{emptyplain}
    {}          % default space above
    {}          % default space below
    {}          % default body font
    {}          % no indent
    {\bfseries} % head font
    {.}         % punctuation after theorem head
    { }         % space after theorem head
    {#3}
\theoremstyle{emptyplain}
\newtheorem*{problem}{}

% Solution Environment
\newenvironment{solution}{
  \begin{proof}[Solution]
    \vspace{-2px}
    \setlength{\parskip}{4px}
    \setlength{\parindent}{0px}
}{
\end{proof}
}


% Opening
\title{Math 2552 Written HW Set 5}
\author{Akash Narayanan}
\date{March 2, 2021}

\begin{document}
  \maketitle

  % Trench 5.1 #11
  \begin{problem}[Trench 5.1.11]
    Find a second solution \(y_{2}\) that isn't a constant multiple of the solution \(y_{1}\).
    Choose \(K\) to conveniently simply \(y_{2}\).
    \begin{equation*}
      y'' - 6y' + 9y = 0; \quad y_{1} = e^{3x}
    \end{equation*}
  \end{problem}

  \begin{solution}
    Given one nontrivial solution of the differential equation, we can find another using Abel's formula.
    Exercise 9 shows that a second solution has the form \(y_{2} = u y_{1}\) where
    \begin{equation*}
      u' = K \frac{e^{-P(x)}}{y_{1}^{2}(x)}
    \end{equation*}
    for arbitrary nonzero constant \(K\).
    For the given differential equation, \(p(x) = -6\) so we have \(P(x) = \int p(x) \; dx = -6x\).
    Then we find
    \begin{align*}
      u' &= K \frac{e^{-6x}}{(e^{3x})^{2}} = K \\
      u &= \int K \; dx = Kx
    \end{align*}
    which shows that \(y_{2} = Kx e^{3x}\).
    We can verify that \(y_{2}\) satisfies the differential equation. Indeed,
    \begin{equation*}
      y_{2}'' - 6y_{2}' + 9y_{2} = (9kx + 6k) e^{3x} - (18kx + 6k) e^{3x} + 9kxe^{3x} = 0
    \end{equation*}
    Certainly \(y_{2}\) is not a constant multiply of \(y_{1}\) since it has a factor of \(x\).
    Setting \(K = 1\) yields a final solution of
    \begin{equation*}
      y_{2} = xe^{3x}
    \end{equation*}
  \end{solution}

  \pagebreak

  % Trench 5.2 #13
  \begin{problem}[Trench 5.2.13]
    Solve the initial value problem.
    \begin{equation*}
      y'' + 14y' + 50y = 0, \quad y(0) = 2, \quad y'(0) = -17
    \end{equation*}
  \end{problem}

  \begin{solution}
    The characterstic polynomial of the differential equation is
    \begin{equation*}
      p(r) = r^{2} + 14r + 50 = (r + 7)^{2} + 1
    \end{equation*}
    The roots of the characterstic polynomial are \(r_{1} = -7 + i\) and \(r_{2} = -7 - i\).
    Then the general solution to the differential equation is
    \begin{equation*}
      y = c_{1} e^{-7x} \cos(x) + c_{2} e^{-7x} \sin(x)
    \end{equation*}
    Differentiating yields
    \begin{equation*}
      y' = -7c_{1} e^{-7x} \cos(x) - c_{1} e^{-7x} \sin(x) - 7c_{2} e^{-7x} \sin(x) + c_{2} e^{-7x} \cos(x)
    \end{equation*}
    From here, we can use the initial conditions to form the system of equations
    \begin{align*}
      2 &= c_{1} \\
      -17 &= -7c_{1} + c_{2}
    \end{align*}
    which has the solution \(c_{1} = 2\) and \(c_{2} = -3\).
    Thus, the solution to the initial value problem is
    \begin{equation*}
      y(x) = 2 e^{-7x} \cos(x) -3 e^{-7x} \sin(x)
    \end{equation*}
  \end{solution}

  \pagebreak

  % Trench 6.1 #7
  \begin{problem}[Trench 6.1.7]
    A weight stretches a spring 1.5 inches in equilibrium.
    The weight is initially displaced 8 inches above equilibrium and given a
    downward velocity of 4 ft/s.
    Find its displacement for \(t > 0\).
  \end{problem}

  \begin{solution}
      The equation of motion is
      \[
      my'' + ky = 0,
      \] 
      or
      \[
      y'' + \frac{k}{m} y = 0.
      \] 
      Given that $mg = k \Delta l$, we have $k / m = g / \Delta l$. Using
      imperial units as in the problem, we have $g = 32 \text{ft/s}^2$.
      Furthermore, we have $\Delta l = 1/8$ ft. Thus, we find
      \[
     \frac{k}{m} = \frac{32}{1/8} = 256 
      \] 
      and the differential equation becomes
      \[
          y'' + 256 y = 0, \quad y(0) = \frac{2}{3}, \quad y'(0) = -4. 
      \] 
      The differential equation has the characteristic equation
      \[
      r^2 + 256 = 0
      \] 
      which has the solutions $r = \pm 16i$. Then the general solution to the
      differential equation is
      \[
      y = c_1 \cos 16t + c_2 \sin 8t.
      \] 
      Differentiating yields
      \[
      y' = -16 c_1 \sin 16t + 16 c_2 \cos 16t.
      \] 
      Using the initial conditions yields the system of equations
      \begin{align*}
          \frac{2}{3} &= c_1 \\
          -4 &= 16 c_2
      \end{align*}
      which has the solution $c_1 = 2/3$ and $c_2 = -1/4$. Thus, the
      displacement is given by
      \[
      y = \frac{2}{3} \cos 16t - \frac{1}{4} \sin 16t.
      \] 
  \end{solution}
\end{document}
