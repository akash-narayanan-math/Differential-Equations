\documentclass[11pt, titlepage]{article}
% Common packages/environments to remove clutter

% Packages
\usepackage[utf8]{inputenc}
\usepackage{amsmath, amsfonts, amssymb, amsthm, enumitem, tikz, import, mathtools}
\usepackage[
  top=2cm,
  bottom=2cm,
  left=3cm,
  right=3cm,
  headheight=17pt,
  includehead, includefoot,
  heightrounded,
]{geometry}

% Problem environment
\newtheoremstyle{emptyplain}
    {}          % default space above
    {}          % default space below
    {}          % default body font
    {}          % no indent
    {\bfseries} % head font
    {.}         % punctuation after theorem head
    { }         % space after theorem head
    {#3}
\theoremstyle{emptyplain}
\newtheorem*{problem}{}

% Solution Environment
\newenvironment{solution}{
  \begin{proof}[Solution]
    \vspace{-2px}
    \setlength{\parskip}{4px}
    \setlength{\parindent}{0px}
}{
\end{proof}
}


% Opening
\title{Math 2552 Written HW Set 7}
\author{Akash Narayanan}
\date{March 23, 2021}

\begin{document}
    \maketitle

    % Trench 3.1 #7
    \begin{problem}[Trench 3.1.7]
        Use Euler's method with step sizes $h = 0.1, h = 0.05$, and $h = 0.025$
        to find approximate values of the solution of the initial value problem
        \[
            y' + \frac{2}{x}y = \frac{3}{x^3} + 1, \quad y(1) = 1
        \] 
        at $x = 1.0, 1.1, 1.2, 1.3, \ldots, 2.0.$ Compare these approximate
        values with the values of the exact solution
        \[
            y = \frac{1}{3x^2}(9 \ln x + x^3 + 2),
        \] 
        which can be obtained by the method of Section 2.1. Present your results
        in a table like Table 3.1.1.
    \end{problem}

    \pagebreak

    % Trench 3.2 #7
    \begin{problem}[Trench 3.2.7]
        Use the improved Euler method with step sizes $h = 0.1, h = 0.05$, and
        $h = 0.025$ to find approximate values of the solution of the initial
        value problem
        \[
            y' + \frac{2}{x}y = \frac{3}{x^3} + 1, \quad y(1) = 1
        \] 
        at $x = 1.0, 1.1, 1.2, 1.3, \ldots, 2.0.$ Compare these approximate
        values with the values of the exact solution
        \[
            y = \frac{1}{3x^2}(9 \ln x + x^3 + 2),
        \] 
        which can be obtained by the method of Section 2.1. Present your results
        in a table like Table 3.2.2.
    \end{problem}

    \pagebreak

    % Trench 3.3 #7
    \begin{problem}[Trench 3.3.7]
        Use the Runge-Kutta method with step sizes $h = 0.1, h = 0.05$, and
        $h = 0.025$ to find approximate values of the solution of the initial
        value problem
        \[
            y' + \frac{2}{x}y = \frac{3}{x^3} + 1, \quad y(1) = 1
        \] 
        at $x = 1.0, 1.1, 1.2, 1.3, \ldots, 2.0.$ Compare these approximate
        values with the values of the exact solution
        \[
            y = \frac{1}{3x^2}(9 \ln x + x^3 + 2),
        \] 
        which can be obtained by the method of Section 2.1. Present your results
        in a table like Table 3.3.1.

    \end{problem}
\end{document}
