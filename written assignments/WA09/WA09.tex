\documentclass[11pt, titlepage]{article}
% Common packages/environments to remove clutter

% Packages
\usepackage[utf8]{inputenc}
\usepackage{amsmath, amsfonts, amssymb, amsthm, enumitem, tikz, import, mathtools}
\usepackage[
  top=2cm,
  bottom=2cm,
  left=3cm,
  right=3cm,
  headheight=17pt,
  includehead, includefoot,
  heightrounded,
]{geometry}

% Problem environment
\newtheoremstyle{emptyplain}
    {}          % default space above
    {}          % default space below
    {}          % default body font
    {}          % no indent
    {\bfseries} % head font
    {.}         % punctuation after theorem head
    { }         % space after theorem head
    {#3}
\theoremstyle{emptyplain}
\newtheorem*{problem}{}

% Solution Environment
\newenvironment{solution}{
  \begin{proof}[Solution]
    \vspace{-2px}
    \setlength{\parskip}{4px}
    \setlength{\parindent}{0px}
}{
\end{proof}
}


% Opening
\title{Math 2552 Written HW Set 9}
\author{Akash Narayanan}
\date{April 6, 2021}

\begin{document}
    \maketitle

    % Trench 8.2 #1a
    \begin{problem}[Trench 8.2.1a]
        Use the table of Laplace transforms to find the inverse Laplace
        transform.
        \[
            \frac{3}{(s-7)^{4}}
        \]
    \end{problem}

    \begin{solution}
        Using the linearity of the inverse Laplace transform and the Shifting
        Theorem, we may write
        \[
            L^{-1} \left( \frac{3}{(s-7)^{4}} \right) = 3 L^{-1} \left(
                \frac{1}{(s-7)^{4}} \right) = \frac{3}{6} e^{7t} L^{-1}
                \left(\frac{6}{s^{4}} \right)
        \] 
        We evaluate the rightmost side with the table of Laplace transforms,
        noting that $L(s)[t^{n}] = n! / s^{n+1}$, and find
        \[
            L^{-1} \left(\frac{3}{(s-7)^{4}} \right) = \frac{e^{7t} t^3}{2}
        \] 

    \end{solution}

    \pagebreak

    % Trench 8.2 #1b
    \begin{problem}[Trench 8.2.1b] 
        Use the table of Laplace transforms to find the inverse Laplace
        transform.
        \[
            \frac{2s - 4}{s^2 - 4s + 13}
        \]
    \end{problem}

    \begin{solution}
        We start by completing the square in the denominator, yielding
        \[
            \frac{2s - 4}{s^2 - 4s + 13} = \frac{2s - 4}{(s - 2)^2 + 9}
        \] 
        Using the shifting theorem and the linearity of the inverse Laplace
        transform, we obtain
        \[
            L^{-1} \left( \frac{2s - 4}{(s - 2)^2 + 9} \right) = 2 e^{2t} L^{-1}
                \left(\frac{s}{s^2 + 9} \right)
        \] 
        Using the table of Laplace transforms, we note that $L(s)[\cos \omega t]
        = s / (s^2 + \omega^2)$. Letting $\omega = 3$, we obtain
        \[
            L^{-1} \left(\frac{2s-4}{s^2-4s+13} \right) = 2e^{2t}\cos(3t)
        \] 
    \end{solution}

    \pagebreak

    % Trench 8.2 #3a
    \begin{problem}[Trench 8.2.3a]
        Use Heaviside's method to find the inverse Laplace transform.
        \[
            \frac{3 - (s + 1)(s - 2)}{(s + 1)(s + 2)(s - 2)}
        \]
    \end{problem}

    \begin{solution}
        We write
        \[
            \frac{3 - (s+1) (s-2)}{(s+1) (s+2) (s-2)} = \frac{A}{s+1} +
            \frac{B}{s+2} + \frac{C}{s-2}
        \] 
        Setting $s = -1$ and ignoring the $(s+1)$ factor in the denominator
        yields
        \[
            A = \frac{3 - (0)(-3)}{(1)(-3)} = -1
        \] 
        Doing the same for the other coefficients gives
        \[
            B = \frac{3 - (-1)(-4)}{(-1)(-4)} = - \frac{1}{4}
        \] 
        and
        \[
            C = \frac{3 - (3)(0)}{(3)(4)} = \frac{1}{4}
        \] 
        Thus, we find
        \begin{align*}
            L^{-1}\left(\frac{3 - (s + 1)(s - 2)}{(s + 1)(s + 2)(s - 2)}\right)
            &= 
            - L^{-1} \left(\frac{1}{s+1}\right) - \frac{1}{4} L^{-1}
            \left(\frac{1}{s+2}\right) + \frac{1}{4} L^{-1}
            \left(\frac{1}{s-2}\right) \\
            &= -e^{-t} - \frac{1}{4} e^{-2t} + \frac{1}{4} e^{2t}
        \end{align*}
    \end{solution}
\end{document}
