\documentclass[11pt, titlepage]{article}
% Common packages/environments to remove clutter

% Packages
\usepackage[utf8]{inputenc}
\usepackage{amsmath, amsfonts, amssymb, amsthm, enumitem, tikz, import, mathtools}
\usepackage[
  top=2cm,
  bottom=2cm,
  left=3cm,
  right=3cm,
  headheight=17pt,
  includehead, includefoot,
  heightrounded,
]{geometry}

% Problem environment
\newtheoremstyle{emptyplain}
    {}          % default space above
    {}          % default space below
    {}          % default body font
    {}          % no indent
    {\bfseries} % head font
    {.}         % punctuation after theorem head
    { }         % space after theorem head
    {#3}
\theoremstyle{emptyplain}
\newtheorem*{problem}{}

% Solution Environment
\newenvironment{solution}{
  \begin{proof}[Solution]
    \vspace{-2px}
    \setlength{\parskip}{4px}
    \setlength{\parindent}{0px}
}{
\end{proof}
}


% Opening
\title{Math 2552 Written HW Set 4}
\author{Akash Narayanan}
\date{February 16, 2021}

\begin{document}
  \maketitle

  % Trench 10.3 #9
  \begin{problem}[Trench 10.3.9]
    Let
    \begin{equation*}
      A =
      \begin{bmatrix*}[r]
        -4 & -10 \\
        3 & 7
      \end{bmatrix*}, \;
      y_{1} =
      \begin{bmatrix*}[r]
        -5e^{2t} \\
        3e^{2t}
      \end{bmatrix*}, \;
      y_{2} =
      \begin{bmatrix*}[r]
        2e^{t} \\
        -e^{t}
      \end{bmatrix*}, \;
      k =
      \begin{bmatrix*}[r]
        -19 \\
        11
      \end{bmatrix*}
    \end{equation*}
    \begin{enumerate}[label={(\alph*)}]
      \item Verify that \(\{y_{1}, y_{2}\}\) is a fundamental set of solutions for \(y' = Ay\).
      \item Solve the initial value problem
      \begin{equation*}
        y' = A y, \; y(0) = k. \tag{A}
      \end{equation*}
      \item Use the result of Exercise 6(b) to find a formula for the solution of (A) for an arbitrary initial vector \(k\).
    \end{enumerate}
  \end{problem}

  \begin{solution}
    It is easy to see that the two vectors satisfy the system of differential equations.
    To show the two are linearly independent, it suffices to show that the Wronskian is non-zero.
    \begin{equation*}
      W(y_{1}, y_{2}) =
      \begin{vmatrix*}[r]
        -5e^{2t} & 2e^{t} \\
        3e^{2t} & -e^{t}
      \end{vmatrix*}
      = -e^{3t} \neq 0
    \end{equation*}
    Thus, \(\{y_{1}, y_{2}\}\) does indeed form a fundamental set of solutions.
    That is, the general solution has the form
    \begin{equation*}
      y = c_{1} y_{1} + c_{2} y_{2}
    \end{equation*}
    For the particular case \(y(0) = k\), we have
    \begin{align*}
      -5 c_{1} +  2c_{2} &= -19 \\
      3 c_{1} - c_{2} &= 11
    \end{align*}
    which yields \(c_{1} = 3\) and \(c_{2} = -2\) for a particular solution of
    \begin{equation*}
      y(t) = 3 y_{1} - 2 y_{2} =
      \begin{bmatrix*}[r]
        -15e^{2t} - 4e^{t} \\
        9e^{2t} + 2e^{t}
      \end{bmatrix*}
    \end{equation*}
    Given the fundamental set \(\{y_{1}, y_{2}\}\), we form the matrices
    \begin{equation*}
      Y(t) =
      \begin{bmatrix*}[r]
        -5e^{2t} & 2e^{t} \\
        3e^{2t} & -e^{t}
      \end{bmatrix*}, \;
      Y^{-1}(t) =
      \begin{bmatrix*}[r]
        e^{-2t} & 2e^{-2t} \\
        3e^{-t} & 5e^{-2t}
      \end{bmatrix*}
    \end{equation*}
    Then for an arbitrary vector \(k\), we have the following solution to (A):
    \begin{equation*}
      y(t) = Y(t) Y^{-1}(0) k =
      \begin{bmatrix*}[r]
        -5e^{2t} & 2e^{t} \\
        3e^{2t} & -e^{t}
      \end{bmatrix*}
      \begin{bmatrix*}[r]
        1 & 2 \\
        3 & 5
      \end{bmatrix*} k
      =
      \begin{bmatrix*}
        -5e^{2t} + 6e^{t} & -10^{2t} + 10e^{t} \\
        3e^{2t} - 3e^{t} & 6e^{2t} - 5e^{t}
      \end{bmatrix*} k
    \end{equation*}
  \end{solution}

  \pagebreak

  % Trench 10.5 #11
  \begin{problem}[Trench 10.5.11]
    Find the general solution.
    \begin{equation*}
      y' =
      \begin{bmatrix*}[r]
        4 & -2 & -2 \\
        -2 & 3 & -1 \\
        2 & -1 & 3
      \end{bmatrix*} y
    \end{equation*}
  \end{problem}

  \begin{solution}
    Denote the matrix by \(A\). Then we compute the eigenvalues of \(A\) by finding the roots of the characteristic polynomial
    \begin{equation*}
      \det(A - \lambda I) = -\lambda^{3} + 10 \lambda^{2} - 32 \lambda + 32
      = -(\lambda - 4)^{2} (\lambda - 2) = 0
    \end{equation*}
    Then the eigenvalues are \(\lambda_{1} = 2\) with multiplicity 1 and \(\lambda_{2} = 4\) with multiplicity 2.
    The eigenvector associated with \(\lambda_{1} = 2\) is
    \begin{equation*}
      \vec{v}_{1} =
      \begin{pmatrix*}[c]
        -2 \\
        -3 \\
        1
      \end{pmatrix*}
    \end{equation*}
    yielding the solution
    \begin{equation*}
      y_{1} = c_{1} e^{2t} \vec{v}_{1}
    \end{equation*}
    The eigenvector associated with \(\lambda_{2} = 4\) is
    \begin{equation*}
      \vec{v}_{2} =
      \begin{pmatrix*}[c]
        0 \\
        -1 \\
        1
      \end{pmatrix*}
    \end{equation*}
    yielding the solution
    \begin{equation*}
      y_{2} = c_{2} e^{4t} \vec{v}_{2}
    \end{equation*}
    A second eigenvector associated with \(\lambda_{2}\) satisfies the equation \((A - 4I) \vec{u} = \vec{v}_{2}\).
    Solving the equation yields
    \begin{equation*}
      \vec{u} =
      \begin{pmatrix*}[c]
        \frac{1}{2} \\
        -a \\
        a
      \end{pmatrix*}
    \end{equation*}
    where \(a\) is free. This forms the solution
    \begin{equation*}
      y_{3} = c_{3} e^{4t} (\vec{u} + t \vec{v}_{2})
    \end{equation*}
    The three solutions are independent and form the general solution
    \begin{equation*}
      y = c_{1} e^{2t}
      \begin{pmatrix*}[c]
        -2 \\
        -3 \\
        1
      \end{pmatrix*} +
      c_{2} e^{4t}
      \begin{pmatrix*}[c]
        0 \\
        -1 \\
        1
      \end{pmatrix*} +
      c_{3} e^{4t}
      \begin{pmatrix*}[c]
        \frac{1}{2} \\
        -a - t \\
        a + t
      \end{pmatrix*}
    \end{equation*}
  \end{solution}

  \pagebreak

  % Trench 10.6 #13
  \begin{problem}[Trench 10.6.13]
    Find the general solution.
    \begin{equation*}
      y' =
      \begin{bmatrix*}[r]
        1 & 1 & 2 \\
        1 & 0 & -1 \\
        -1 & -2 & -1
      \end{bmatrix*} y
    \end{equation*}
  \end{problem}

  \begin{solution}
    We find the eigenvalues by calculating the roots of the characterstic polynomial.
    \begin{equation*}
      \det(A - \lambda I) = \lambda^{3} + 2 \lambda - 4 = -(\lambda + 2) (\lambda - (1 - i)) (\lambda - (1 + i)) = 0
    \end{equation*}
    The eigenvalues are \(\lambda_{1} = -2\), \(\lambda_{2} = 1 - i\), and \(\lambda_{3} = 1 + i\).
    The eigenvector associated with \(\lambda_{1} = -2\) is
    \begin{equation*}
      \vec{v}_{1} =
      \begin{pmatrix*}[c]
        -1 \\
        1 \\
        1
      \end{pmatrix*}
    \end{equation*}
    which yields the solution
    \begin{equation*}
      y_{1} = c_{1} e^{-2t} \vec{v}_{1}
    \end{equation*}
    The eigenvector associated with \(\lambda_{2} = 1 - i\) is
    \begin{equation*}
      v_{2} =
      \begin{pmatrix*}[c]
        i \\
        -1 \\
        1
      \end{pmatrix*}
    \end{equation*}
    This eigenvalue-vector pair yields two solution vectors to the original system, namely
    \begin{equation*}
      w_{1} = c_{2} e^{t}
      \begin{pmatrix*}[c]
        \sin(t) \\
        -\cos(t) \\
        \cos(t)
      \end{pmatrix*}, \quad
      w_{2} = c_{3} e^{t}
      \begin{pmatrix*}[c]
        -\cos(t) \\
        -\sin(t) \\
        \sin(t)
      \end{pmatrix*}
    \end{equation*}
    Thus, the general solution to the system is
    \begin{equation*}
      y = c_{1} e^{-2t}
      \begin{pmatrix*}[c]
        -1 \\
        1 \\
        1
      \end{pmatrix*} +
      c_{2} e^{t}
      \begin{pmatrix*}[c]
        \sin(t) \\
        -\cos(t) \\
        \cos(t)
      \end{pmatrix*} +
      c_{3} e^{t}
      \begin{pmatrix*}[c]
        -\cos(t) \\
        -\sin(t) \\
        \sin(t)
      \end{pmatrix*}
    \end{equation*}
  \end{solution}
\end{document}
