\documentclass[11pt, titlepage]{article}
% Common packages/environments to remove clutter

% Packages
\usepackage[utf8]{inputenc}
\usepackage{amsmath, amsfonts, amssymb, amsthm, enumitem, tikz, import, mathtools}
\usepackage[
  top=2cm,
  bottom=2cm,
  left=3cm,
  right=3cm,
  headheight=17pt,
  includehead, includefoot,
  heightrounded,
]{geometry}

% Problem environment
\newtheoremstyle{emptyplain}
    {}          % default space above
    {}          % default space below
    {}          % default body font
    {}          % no indent
    {\bfseries} % head font
    {.}         % punctuation after theorem head
    { }         % space after theorem head
    {#3}
\theoremstyle{emptyplain}
\newtheorem*{problem}{}

% Solution Environment
\newenvironment{solution}{
  \begin{proof}[Solution]
    \vspace{-2px}
    \setlength{\parskip}{4px}
    \setlength{\parindent}{0px}
}{
\end{proof}
}

\usepackage{fancyhdr}
\usepackage[
    top=2cm,
    bottom=2cm,
    left=3cm,
    right=3cm,
    headheight=17pt,
    includehead, includefoot, heightrounded,
]{geometry}

% Header
\pagestyle{fancy}
\fancyhf{}
\lhead{Math 2552 Midterm 2}
\rhead{Akash Narayanan}

% Opening
\title{Math 2552 Midterm 2}
\author{Akash Narayanan}
\date{March 11, 2021}

\begin{document}
    \maketitle

    \begin{enumerate}
        % Problem 1
        \item (2 points) Determine all values of $\alpha$, if any, for which all
            solutions tend to zero as $t \to \infty$.
            \[
                y'' + (\alpha + 2) y' + 4y = 0, \quad \alpha \in \mathbb{R}
            \] 
            Show your work.

            \begin{solution}
                Consider the characteristic equation $r^2 + (\alpha + 2) + 4 =
                0$. Using the quadratic formula, it has the following roots:
                \begin{align*}
                    r &= \frac{-(\alpha + 2) \pm \sqrt{(\alpha + 2)^2 - 16}}{2}
                    \\
                      &= -(1 + \frac{\alpha}{2}) \pm \frac{1}{2} \sqrt{\alpha^2
                      + 4\alpha - 12}
                \end{align*}
                We now treat separate cases based on the value of the sign of
                the discriminant $d = \alpha^2 + 4\alpha - 12$.

                First consider the case $d < 0$. This occurs when $(\alpha +
                2)^2 - 16 < 0$, or
                \begin{gather*}
                    (\alpha + 2)^2 < 16 \\
                    -4 < \alpha + 2 < 4 \\
                    -6 < \alpha < 2
                \end{gather*}
                For solutions to converge to zero under this condition, the real
                part of the roots must be negative. That is, 
                \begin{gather*}
                    -(1 + \frac{\alpha}{2}) < 0 \\
                    2 + \alpha > 0 \\
                    \alpha > -2
                \end{gather*}
                Therefore, solutions converge to zero for $\alpha \in (-2, 2)$.

                Now consider the case $d = 0$, which occurs when $\alpha = 2$.
                Then the root $r = -2$. Since this is the only root, the
                solution to the differential equation is
                 \[
                y = c_1 e^{-2t} + c_2 t e^{-2t}
                \] 
                which clearly converges to zero as $t$ goes to infinity.

                Finally, consider the case $d > 0$. This occurs when $(\alpha +
                2)^2 - 16 > 0$, or
                \begin{gather*}
                    (\alpha + 2)^2 > 16 \\
                    |\alpha + 2| > 4 \\
                    \alpha + 2 > 4 \quad \text{or} \quad -\alpha - 2 > 4 \\
                    \alpha > 2 \quad \text{or} \quad \alpha < -6
                \end{gather*}
                Note that if $\alpha < -2$ then $r > 0$ for at least one of the
                roots (because the vertex is positive and there is one root
                greater than the vertex) so at least one of the solutions will
                diverge as $t$ goes to infinity.
                Thus, we can restrict our attention to $\alpha >
                2$. We require that the root closest to zero still be negative.
                As an inequality, we obtain
                 \begin{gather*}
                    -\frac{\alpha}{2}-1 + \frac{1}{2}\sqrt{\alpha^2 + 4\alpha -
                    12} < 0 \\
                    \sqrt{\alpha^2 + 4\alpha -12} < \alpha + 2 \\
                    \alpha^2 + 4\alpha - 12 < \alpha^2 + 4\alpha + 4 \\
                    0 < 16
                \end{gather*}
                Since the inequality holds trivially, we have that both roots of
                the characteristic equation are negative if $\alpha > 2$. 
                Thus, all solutions to the differential equation converge to
                zero as $t$ goes to infinity if and only if $\alpha > -2$.
            \end{solution}
        \pagebreak

        % Problem 2
        \item (2 points) Determine whether the functions $y_1 = e^{2t}$ and $y_2
            = e^{3t}$ are a fundamental set of solutions to the differential
            equation $y'' + 5y' + 6y=0$ with $y = y(t)$. Show your work.

            \begin{solution}
                We can check by first seeing if the functions satisfy the
                differential equation. We find
                \[
                y_1'' + 5y_1' + 6y_1 = 4e^{2t} + 10e^{2t} + 6e^{2t} \neq 0
                \] 
                Since $y_1$ does not satisfy the differential equation, the set
                $\{y_1, y_2\}$ is not a fundamental set of solutions.
            \end{solution}
        \pagebreak

        % Problem 3
        \item (1 point) State whether the following differential equation is
            linear or non-linear, and whether it is homogeneous or
            non-homogeneous.
            \[
                y'' + (1-y)y' + 2y = 1
            \] 
            \begin{solution}
                The differential equation is non-linear and non-homogeneous. It
                is non-linear because there is a term $y y'$ which is not a
                linear combination of $y$ and its derivatives. It is
                non-homogeneous because there is a non-zero constant term,
                namely $1$.
            \end{solution}
        \pagebreak

        % Problem 4
        \item (3 points) Construct an initial value problem for the following
            situation. Show your work.

            A 0.05 Newtons (N) force stretches a spring 0.01 m. A mass weighing
            2 kg is attached to the spring, and the spring is also attached to a
            viscous damper that applies a force of 0.4 N when the velocity of
            the mass is 0.1 m/s. The mass is pulled down 0.1 m below its
            equilibrium position and given an initial upward velocity of 0.6 m/s.

            \begin{solution}
                By Newton's Laws, we have $F = kL$.
                Substituting, we find $0.05 = 0.01 k$ so $k = 5$. Similarly, the
                damping force is given by the equation $F_d = \gamma v$.
                Substituting the corresponding values yields $0.4 = 0.1 \gamma$
                so $\gamma = 4$. Given the initial conditions, the corresponding
                initial value problem is
                \[
                    2y'' + 4y' + 5 = 0, \quad y(0) = 0.1, \quad y'(0) = -4
                \] 
            \end{solution}
        \pagebreak

        % Problem 5
        \item (3 points) If $Y = W[f, g]$ is the Wronskian of $f$ and $g$, and
            $u = 3f + g$, $v = f - 3g$, express the Wronskian $W[u, v]$ of $u$
            and $v$ in terms of $Y$. Show your work.

            \begin{solution}
                First, note that $Y = fg' - f'g$. We find that
                \begin{align*}
                    W[u, v] &= uv' - u'v \\
                            &= (3f+g)(f'-3g') - (3f'+g')(f-3g) \\
                            &= (3ff' - 9fg' + gf' - 3gg' - 3ff' + 9f'g + fg' +
                            3gg' \\
                            &= -8(fg' - f'g) \\
                            &= -8Y
                \end{align*}
            \end{solution}
        \pagebreak

        % Problem 6
        \item (3 points) Determine a suitable form for the particular solution
            $Y(t)$ if the method of undetermined coefficients is to be used.
            Please show your work.
            \[
                y'' + 11y' + 24y = 4 \sin(4t) + 2e^{-8t}
            \] 
            \begin{solution}
                We start by considering the general solution to the homogeneous
                equation:
                \[
                y_h'' + 11y_h' + 24y_h = 0
                \] 
                The characteristic equation is $r^2 + 11r + 24 = 0$ which has
                solutions $r_1 = -8$ and $r_2 = -3$. Then the solution to the
                homogeneous equation is
                \[
                y_h = c_1 e^{-8t} + c_2 e^{-3t}
                \] 
                Through inspection, one might guess that a particular solution
                would have the form
                \[
                    Y(t) = a_1 \sin(4t) + a_2 \cos(4t) + a_3 e^{-8t}
                \] 
                However, note that $e^{-8t}$ is a solution to the homogeneous
                equation. We remedy this by including a factor of $t$.
                Therefore, a suitable form for a particular solution using the
                method of undetermined coefficients is
                \[
                    Y(t) = a_1 \sin(4t) + a_2 \cos(4t) + a_3 t e^{-8t}
                \] 
            \end{solution}
        \pagebreak

        % Problem 7
        \item (4 points) The position of a moving object, $y(t)$, for time $t
            \geq 0$ satisfies the IVP
            \[
                y'' + 2y' + 4y = 0, \quad y(0) = 1, \; y'(0) = 4, \quad y = y(t)
            \] 
            \begin{enumerate}[label={(\alph*)}]
            \item Express the differential equation in the IVP as a first-order
                system in the form $\vec{x}' = A \vec{x}$.
            \item Solve the DE using any method you like. You do not need to
                solve the IVP, but show your work.
            \item Sketch the trajectory of the object for $t \geq 0$ in the
                phase plane. Indicate the location of the object at time $t =
                0$, the direction of motion, and label your axes.
            \end{enumerate}

            \begin{solution}
                Let $x_1 = y$ and $x_2 = y'$. Then we find $x_1' = y' = x_2$ and
                $x_2' = -4y - 2y' = -4x_1 - 2x_2$. In matrix form, this can be
                expressed as
                \[
                    \vec{x}' = 
                    \begin{pmatrix}
                        0 & 1 \\
                        -4 & -2
                    \end{pmatrix} \vec{x}, \quad \vec{x}(0) =
                    \begin{pmatrix}
                        1 \\
                        4
                    \end{pmatrix}
                \] 
                To solve the homogeneous differential equation, consider the
                characteristic equation $r^2 + 2r + 4 = 0$ which has solutions
                $r = -1 \pm i\sqrt{3}$. Then the general solution is
                \[
                    y = e^{-t} \left(c_1 \cos(\sqrt{3} t) + c_2 \sin(\sqrt{3} t)
                    \right)
                \] 

                \begin{figure}[h]
                    \centering
                    \def\svgwidth{0.7\columnwidth}
                    \import{media/}{phasePortrait7.pdf_tex}
                \end{figure}
            \end{solution}
        \pagebreak

        % Problem 8
        \item (10 points) Use the variation of parameters method to identify the
            general solution to
            \[
                \vec{x}' = 
                \begin{pmatrix}
                    2 & 4 \\
                    4 & 2
                \end{pmatrix} \vec{x} +
                \begin{pmatrix}
                    12 \\
                    0
                \end{pmatrix}
            \]
            \begin{solution}
                Let $A$ be the coefficient matrix of $\vec{x}$. Then
                \[
                    \det(A - \lambda I) = \lambda^2 - 4\lambda - 12 = (\lambda +
                    2) (\lambda - 6) = 0
                \] 
                and the eigenvalues of the system are $\lambda_1 = -2$ and
                $\lambda_2 = 6$. The corresponding eigenvectors can be found by
                solving the equation $(A - \lambda I) \vec{v} = 0$. We have
                \[
                    (A + 2I) \vec{v}_1 =
                    \begin{pmatrix}
                        4 & 4 \\
                        4 & 4
                    \end{pmatrix} \vec{v}_1 = 0 \Longrightarrow
                    4x_1 + 4x_2 = 0 \Longrightarrow
                    x_1 = -x_2
                \] 
                Letting $x_2 = -1$, we find the eigenvector corresponding to
                $\lambda_1 = -2$ to be
                \[
                    \vec{v}_1 =
                    \begin{pmatrix}
                        1 \\
                        -1
                    \end{pmatrix}
                \] 
                Solving for $\vec{v}_2$, we find
                \[
                    (A - 6I) \vec{v}_2 = 
                    \begin{pmatrix}
                        -4 & 4 \\
                        4 & -4
                    \end{pmatrix} \vec{v}_2 = 0 \Longrightarrow
                    4x_1 - 4x_2 = 0 \Longrightarrow
                    x_1 = x_2
                \] 
                Letting $x_2 = 1$, we find the eigenvector corresponding to
                $\lambda_2 = 6$ to be
                \[
                    \vec{v}_2 = 
                    \begin{pmatrix}
                        1 \\
                        1
                    \end{pmatrix}
                \] 
                Then the solution to the homogeneous system is
                \[
                y_h = c_1 e^{-2t}
                \begin{pmatrix}
                    1 \\
                    -1
                \end{pmatrix} + c_2 e^{6t}
                \begin{pmatrix}
                    1 \\
                    1
                \end{pmatrix}
                \] 
                The corresponding fundamental matrix is
                \[
                X = 
                \begin{pmatrix}
                    e^{-2t} & e^{6t} \\
                    -e^{-2t} & e^{6t}
                \end{pmatrix}
                \] 
                We calculate
                \begin{gather*}
                    \det(X) = e^{4t} + e^{4t} = 2e^{4t} \\
                    X^{-1} = \frac{1}{\det(X)}
                    \begin{pmatrix}
                        e^{6t} & -e^{6t} \\
                        e^{-2t} & e^{-2t}
                    \end{pmatrix} = \frac{1}{2}
                    \begin{pmatrix}
                        e^{2t} & -e^{2t} \\
                        e^{-6t} & e^{-6t} 
                    \end{pmatrix}
                \end{gather*}
                A particular solution is
                \[
                y_p = X \int X^{-1} g \, dt
                \] 
                where $g$ is the non-homogeneous part of the system of
                differential equations. Calculating, we find
                \begin{align*}
                    y_p &= \frac{1}{2} X \int
                    \begin{pmatrix}
                        e^{2t} & -e^{2t} \\
                        e^{-6t} & e^{-6t} 
                    \end{pmatrix}
                    \begin{pmatrix}
                        12 \\
                        0
                    \end{pmatrix} \, dt \\
                        &= \frac{1}{2} X \int
                    \begin{pmatrix}
                        12e^{2t} \\
                        12e^{-6t}
                    \end{pmatrix} \, dt \\
                        &= \frac{1}{2}
                    \begin{pmatrix}
                        e^{-2t} & e^{6t} \\
                        -e^{-2t} & e^{6t}
                    \end{pmatrix}
                    \begin{pmatrix}
                        6e^{2t} \\
                        -2e^{-6t}
                    \end{pmatrix} \\
                        &=
                    \begin{pmatrix}
                        2 \\
                        -4
                    \end{pmatrix}
                \end{align*}
                Then the general solution to the system of differential
                equations is
                \[
                    \vec{x} = c_1 e^{-2t}
                    \begin{pmatrix}
                        1 \\
                        -1
                    \end{pmatrix} + c_2 e^{6t}
                    \begin{pmatrix}
                        1 \\
                        1
                    \end{pmatrix} +
                    \begin{pmatrix}
                        2 \\
                        -4
                    \end{pmatrix}
                \] 
            \end{solution}

        \pagebreak

        % Problem 9
        \item (10 points) Solve the DE. Show your work.
            \[
                y'' + 25y = 60 \cos(5t), \quad y = y(t)
            \] 

            \begin{solution}
                We first consider the homogeneous equation
                \[
                y_h'' + 25y_h = 0
                \] 
                The characteristic equation is $r^2 + 25 = 0$ which has
                solutions $r = \pm 5i$. Then the general form for the
                homogeneous solution is
                \[
                    y_h = c_1 \cos(5t) + c_2 \sin(5t)
                \] 
                We find a particular solution using the method of undetermined
                coefficients. Assume that a particular solution has the form
                \[
                    y_p = At \cos(5t) + Bt \sin(5t)
                \] 
                Then we have
                \begin{gather*}
                    y_p' = (5Bt + A) \cos(5t) + (B - 5At) \sin(5t) \\
                    y_p'' = (10B - 25At) \cos(5t) - (25Bt + 10A) \sin(5t)
                \end{gather*}
                Substituting this into the differential equation, we obtain
                \begin{align*}
                    y_p'' + 25y_p &= (10B - 25At) \cos(5t) - (25Bt + 10A)
                    \sin(5t) + 25(At \cos(5t) + Bt \sin(5t)) \\
                                  &= 10B \cos(5t) - 10A \sin(5t) = 60 \cos(5t)
                \end{align*}
                That is, $A = 0$ and $B = 6$ so a particular solution is
                \[
                    y_p = 6t \sin(5t)
                \] 
                Thus, the general solution to the differential equation is
                \[
                    y(t) = c_1 \cos(5t) + c_2 \sin(5t) + 6t \sin(5t)
                \] 
            \end{solution}
        \pagebreak

        % Problem 10
        \item (10 points) Solve the DE using variation of parameters. Solutions
            to the homogeneous problem are $y_1 = t^2$ and $y_2 = t^{-2}$.
            Please show your work.
            \[
                t^2 y'' + ty' - 4y = t, \quad y = y(t), \quad t > 0
            \] 
            \begin{solution}
                We start by rewriting the differential equation in standard
                form:
                \[
                y'' + \frac{1}{t}y' - \frac{4}{t^2}y = \frac{1}{t}
                \] 
                Given the fundamental set of solutions, we can construct a
                particular solution of the form
                \[
                y_p = u_1 y_1 + u_2 y_2
                \] 
                for functions $u_1$ and $u_2$.
                We first calculate the Wronskian
                \begin{align*}
                    W[y_1, y_2] &= y_1 y_2' - y_1' y_2 \\
                              &= t^2 (-2t^{-3}) - (2t)(t^{-2}) \\
                                &= -4t^{-1}
                \end{align*}
                Then we have
                \begin{gather*}
                    u_1'(t) = \frac{-y_2(t) g(t)}{W[y_1, y_2]} = \frac{-t^{-2}
                    \cdot t^{-1}}{-4t^{-1}} = \frac{1}{4t^2} \\
                    u_2'(t) = \frac{y_1(t) g(t)}{W[y_1, y_2]} = \frac{t^2 \cdot
                    t^{-1}}{-4t^{-1}} = -\frac{t^2}{4}
                \end{gather*}
                where $g(t)$ is the non-homogeneous part of the standard form
                differential equation. Integrating with respect to $t$ yields
                \begin{gather*}
                    u_1 = \int \frac{1}{4t^2} \, dt = -\frac{1}{4t} \\
                    u_2 = \int -\frac{t^2}{4} \, dt = -\frac{t^3}{12}
                \end{gather*}
                Then a particular solution to the differential equation is
                \begin{align*}
                    y_p &= -\frac{1}{4t} t^2 - \frac{t^3}{12} t^{-2} \\
                        &= -\frac{t}{3}
                \end{align*}
                Finally, the general solution to the differential equation is
                \[
                    y(t) = c_1 t^2 + c_2 t^{-2} - \frac{t}{3}
                \] 
            \end{solution}
    \end{enumerate}
\end{document}
